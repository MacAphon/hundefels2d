\documentclass[a4paper,titlepage]{article}
\title{CT Projekt: Raycasting engine (Hundefels 2D)}
\author{Christian Korn}
\date{20.10.2021 - 11.01.2022}

\usepackage[ngerman]{babel}
\usepackage{graphicx}

\begin{document}
\maketitle
\tableofcontents

\newpage

\section{Ziele}

\subsection{Muss-Ziele}
Wenn diese Ziele nicht erreicht werden, wird das Projekt als Fehlschlag angesehen.

\begin{itemize}
\item Anzeigen eines 2D Levels in 2,5D (Raycasting Methode)
\item Bewegungsfreiheit im Level (Translation und Rotation)
\end{itemize}

\subsection{Soll-Ziele}
Diese Ziele müssen nicht unbedingt erreicht werden, sind aber für einen vollen Erfolg nötig.

\begin{itemize}
\item Laden von Leveln aus Dateien
\item Anzeigen von anderen Objekten im Level (z.B. Gegner, Items)
\item Kollisionserkennung
\end{itemize}

\subsection{Kann-Ziele}
Diese Ziele sind nicht nötig, können aber nach Vollendung der Höheren Ziele in Angriff genommen werden.

\begin{itemize}
\item Gegner KI
\item Schießen
\item Sprites
\item Texturen für Wände
\item visuelle Effekte (view bobbing, Blutspritzer)
\end{itemize}

\newpage

\section{Verwendete Technologien}

\subsection{Python}

Das Projekt wurde mit Python \verb|3.9.7| erstellt, müsste aber auch in späteren Versionen funktionieren.

Die einzige verwendete Python-Bibliothek, die nicht Teil der Standartbibliothek ist, ist Pygame. Sie kann mit ``\verb|python3 -m pip install -U pygame --user|'' installiert werden.

\subsection{Dokumentation}

Die Projektdokumentation wurde mit \LaTeX erstellt,
UML Klassendiagramme wurden mit YUML erstellt.
\newpage

\section{Mathematische Funktionsweise}

Die Berechnungen werden 1 mal pro Frame ausgeführt. Idealerweise heißt das, dass sie 60 mal pro Sekunde erfolgen. wenn die Rechenleistung nicht ausreicht wird eine Warnung angezeigt.

\subsection{Bewegung}

Der aktuelle Bewegungszustand und die Position werden in den Variablen \verb|_state| und \verb|_position| gespeichert.

\subsubsection*{Drehung}


\subsubsection*{Laufen}

\subsection{Raycasting}
\setlength{\unitlength}{1cm}
\begin{picture}(5,4)
	\multiput(0,0)(1,0){6}{\line(0,1){4}}
	\multiput(0,0)(0,1){5}{\line(1,0){5}}
	\multiput(0,0)(0.1,0){50}{\line(0,1){1}}
	\put(1,3){\circle*{0.3}}
	\thicklines
	\put(1,3){\vector(3,-2){3}}
\end{picture}


\newpage

\section{Programmaufbau}
\includegraphics[scale=0.35]{./img/yuml1}

\newpage

\section{Steuerung}

\subsection{Command-Line Argumente}

\begin{itemize}
	\item \verb|-h --help| zeigt die CLI Argumente und beendet das Programm.
	\item \verb|-l --level| lädt das angegebene Level oder die angegebene Level Datei.
	\item \verb|--fov| ändert den Blickwinkel (angegeben in Grad) Standartwert ist 90°.
	\item \verb|--rays| ändert die horizontale Auflösung (Anzahl der gesendeten Strahlen) Standartwert ist 90. Höhere Werte können die Leistung beeinträchtigen.
\end{itemize}

\subsection{Levelerstellung}
Level werden im JSON-Format gespeichert.
\begin{itemize}
	\item \verb|"map": [[int]]| Die Map: 1 entspricht einer Wand, 0 Leerraum. Die Map muss quadratisch sein (ansonsten crasht das Programm)
	\item \verb|"size": int| Die Größe der Map. Muss dem tatsächlichen Wert entsprechen.
	\item \verb|"start_pos": [x: int, y: int, r: int]| Die Startposition des Spielers. \verb|x| und \verb|y| sind Werte zwischen 0 und 512, sie geben die Position in Pixeln an. \verb|r| ist zwischen 0 und 360 und ist die Drehung in Grad.
	\item \verb|"entities": [], "enemies": []| Enthalten aktuell keine Werte und werden für zukünftigen Gebrauch freigehalten.
	
\end{itemize}

\subsection{Bewegung}

\subsubsection*{Translation (Laufen)}
\begin{itemize}
\item Vorwärts: `W'
\item Links: `A'
\item Rückwärts: `S'
\item Rechts: `D'
\end{itemize}

\subsubsection*{Rotation}
\begin{itemize}
\item Links: linke Pfeiltaste ($\leftarrow$)
\item Rechts: rechte Pfeiltaste ($\rightarrow$)
\end{itemize}

\subsection{UI}
\includegraphics[scale=1.11]{./img/ui}\\

Das Anzeigefenster ist 1024 auf 512 Pixel groß.

Die linke Hälfte enthält die Kartenansicht. Dabei ist weiß ein Wandblock und schwarz leer. Die  Karte enthält auch den Spieler: ein gelber Kreis mit einer Linie um die Blickrichtung anzuzeigen. Die grünen Strahlen, die vom Spieler ausgehen repräsentieren die im Hintergrund berechneten Strahlen und damit das Blickfeld des Spielers.

Die rechte Hälfte des Fensters ist der First-Person Viewport: Die untere Hälfte ist grau, die Obere hellblau. dazwischen sind die Wände in rot. Vertikale (in y-Richtung verlaufende) Wände sind dunkelrot, horizontale Wände Hellrot. Es können teilweise Streifen der anderen Wandfarbe gesehen werden, das kann aktuell leider nicht behoben werden.

\newpage

\begin{flushleft}
\begin{thebibliography}{99}
	
\bibitem{3dsage} 3DSage: ~ ``Make Your Own Raycaster Part 1'' ~ https://youtu.be/gYRrGTC7GtA\\ Quellcode verfügbar unter https://github.com/3DSage/OpenGL-Raycaster\_v1

\bibitem{pygametut} Pygame tutorial: 	https://www.pygame.org/docs/tut/MakeGames.html
\end{thebibliography}
\end{flushleft}

\end{document}